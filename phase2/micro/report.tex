\documentclass[12pt]{report}
\usepackage[utf8]{inputenc}
\usepackage{cite}
\usepackage[hidelinks]{hyperref}
\usepackage{graphicx}
\usepackage{amsfonts}
\usepackage{mathtools}
\usepackage{caption}
\usepackage{fancyhdr}
\usepackage{multirow}

\pagestyle{fancy}
\fancyhf{}
\rhead{Micro Modules}
\lhead{\thesection}
\rfoot{\thepage}

\DeclarePairedDelimiter\ceil{\lceil}{\rceil}
\DeclarePairedDelimiter\floor{\lfloor}{\rfloor}

\title{\textbf{Micro Modules Design Report}\\Team \#5}
\author{
  Mohamed Shawky\\
  \small\texttt{SEC:2, BN:16}
  \and
  Remonda Talaat\\
  \small\texttt{SEC:1, BN:20}
  \and
  Evram Youssef\\
  \small\texttt{SEC:1, BN:9}
  \and
  Mahmoud Adas\\
  \small\texttt{SEC:2, BN:21}
  \and
  Reham Gamal\\
  \small\texttt{SEC:1, BN:21}
  \and
  Mazen Amr\\
  \small\texttt{SEC:2, BN:8}
  \and
  Mohamed Ahmed Ibrahim\\
  \small\texttt{SEC:2, BN:9}
  \and
  Mahmoud Mohamed\\
  \small\texttt{SEC:2, BN:22}
}
\date{\today}

\begin{document}
    \thispagestyle{empty}

    \maketitle
    \tableofcontents
    \clearpage

    \pagenumbering{arabic}

    \section{Introduction}
    This report documents some of the implementation details in the micro modules, plus it contains the synthesis statistics for those modules. 

    \section{FPU (Float/Fixed Point Unit)}
    The fpu is a component that mux-es between an adder/subtractor, multiplier and divider, all of which operate only on floats/fixed point numbers. 

    The fpu itself is not instantited in the project, but rather its subunits. 

    All fpu subunits follow the fpu specs (including all ports except `operation` and all logic and state) in `0.14.4` in the main document. 

    FPU is implemented in `src/fpu.vhdl` , while its subunits are implemented in `src/fpu\_subunits` . 

    Each fpu subunit has multiple architectures. 
    They are: 
    \begin{itemize}
      \item `with\_operators` : copied from phase1, used temporarily in macro modules while other architectures are developed. 
    \item `first\_algo` . 
    \item `sec\_algo` : divider does have only first\_algo and with\_operators. 
    \end{itemize}

    \subsection{FPU Adder/Subtractor}
    \subsubsection{Synthesis Statistics}
    \begin{tabular}{||c|c|c|c||}
      \hline
      Architecture & Area & Power & Time\\\hline\hline
      first algo & cell=255 Area per cell =322 & 1013.76 & 1201\\
      sec algo & cell=321 Area per cell =376 & 830.51 & 1005 \\\hline
  \end{tabular}

  \subsection{FPU Multiplier}
    \subsubsection{Synthesis Statistics}
    \begin{tabular}{||c|c|c|c||}
      \hline
      Architecture & Area & Power & Time\\\hline\hline
      sec algo & 1745 & 3186.622559 & 59739.5\\\hline
  \end{tabular}

  \subsection{FPU Divider}
    \subsubsection{Synthesis Statistics}
    \begin{tabular}{||c|c|c|c||}
      \hline
      Architecture & Area & Power & Time\\\hline\hline
      first algo & 3445 & 3557.846191 & 57189.6\\\hline
  \end{tabular}

  \section{RAM}
  RAM is generic in:

  \begin{itemize}
    \item number of words
    \item width of address bus
    \item width of data bus
  \end{itemize}

  RAM has seperate bus for read and write, and performas both on falling edge of the clock. 

  RAM also has asyncrounous reset signal that performs parallel load into all internal lateches. 

  \subsection{Synthesis Statistics}
  \begin{tabular}{||c|c|c||}
      \hline
      Area & Power & Time\\\hline\hline
      cells=8910 Area per cell=26279 & 4437.828 & 122.4\\\hline
  \end{tabular}

  \section{CPU}
  Composed of 3 parts:
  \begin{enumerate}
    \item `scripts/preprocessor.py` : reads input json and outputs `.in` file that contains ASCII 0s and 1s in lines, each line is 32 byte in width. 
    \item `test/cpu\_tb.vhdl` : testbench that reads `.in` files in `input` directory and performs the main simulation by feeding the compressed data and switching state input and receiving output data and writing them back to `out/*.out` files. 

    Some lines in `cpu\_tb.vhdl` are commented for this phase, because they are integration points for next phases. 
    \item `scripts/output\_fromatter.py` : convert ASCII output (0s and 1s) from `cpu\_tb.vhdl` into human readable json. 
  \end{enumerate}
\end{document}